\documentclass[10pt]{article}

% =========================
% Preamble (matching existing templates)
% =========================
\usepackage[english]{babel}
\usepackage[utf8]{inputenc}
\usepackage[T1]{fontenc}
\usepackage{amsmath}
\usepackage{amssymb}
\usepackage{amsfonts}
\usepackage{lmodern}
\usepackage{xcolor}
\usepackage{geometry}
\usepackage{tcolorbox}
\tcbuselibrary{raster, skins} % Removed breakable. Skins required for 'attach boxed title'.
\usepackage{hyperref}
\hypersetup{colorlinks=true, linkcolor=blue}

% Geometry: Maximize card space
\geometry{a4paper, portrait, margin=8mm}

\setlength{\parindent}{0pt}
\setlength{\parskip}{0.2em}

% Colors (Knowledge Graph Theme)
\definecolor{cardblue}{RGB}{0, 80, 158}
\definecolor{cardborder}{RGB}{100, 100, 100}
\definecolor{linkpurple}{RGB}{120, 50, 150}
\definecolor{taggreen}{RGB}{0, 120, 60}
\definecolor{defgreen}{RGB}{61,176,0}
\definecolor{thmred}{RGB}{180, 0, 0}

\pagenumbering{gobble}

% Card raster configuration
\tcbset{
    zettelstyle/.style={
        enhanced,
        sharp corners,
        colback=white,
        colframe=cardblue,
        boxrule=0.8pt,
        fonttitle=\bfseries\small,
        coltitle=white,
        colbacktitle=cardblue,
        attach boxed title to top left={xshift=3mm, yshift=-2mm},
        boxed title style={sharp corners, colframe=cardblue},
        top=5mm,
        bottom=2mm,
        left=2mm,
        right=2mm,
    }
}

% Connection macro
\newcommand{\conn}[1]{\textcolor{linkpurple}{\small $\rightarrow$ \textit{#1}}}

% Tag macro
\newcommand{\tags}[1]{\par\vfill\textcolor{taggreen}{\scriptsize\ttfamily #1}}

\begin{document}

{\Large\bfseries\color{cardblue} Graph Theory --- Zettelkasten Cards \hfill \normalsize Knowledge Atoms}
\vspace{3mm}\hrule\vspace{5mm}

%==========================================================
% PAGE 1: 6 cards
%==========================================================
\begin{tcbitemize}[
    raster columns=2,
    raster rows=3,
    raster height=0.92\textheight,
    raster equal height=rows,
    raster column skip=4mm,
    raster row skip=4mm,
]

\tcbitem[zettelstyle, title={\texttt{G-01} --- Graph Definition}]
\footnotesize
\textcolor{defgreen}{\textbf{Def.}} A \textbf{graph} $G = (V, E)$ consists of:
\begin{itemize}
\item $V$: finite set of \textbf{vertices} (nodes)
\item $E \subseteq V \times V$: set of \textbf{edges}
\end{itemize}
\textbf{Undirected:} $(u,v) = (v,u)$\\
\textbf{Directed (digraph):} $(u,v) \neq (v,u)$

\conn{See G-02: Degree}\\
\conn{See G-05: Adjacency Matrix}
\tags{\#foundations \#definitions}

\tcbitem[zettelstyle, title={\texttt{G-02} --- Degree}]
\footnotesize
\textcolor{defgreen}{\textbf{Def.}} The \textbf{degree} $\deg(v)$ of vertex $v$ is the number of edges incident to $v$.

\textcolor{thmred}{\textbf{Thm.}} (Handshaking Lemma)
\[
\sum_{v \in V} \deg(v) = 2|E|
\]
\textbf{Corollary:} Number of odd-degree vertices is even.

\conn{See G-01: Graph Definition}\\
\conn{See G-04: Eulerian Graphs}
\tags{\#degree \#handshaking}

\tcbitem[zettelstyle, title={\texttt{G-03} --- Paths \& Cycles}]
\footnotesize
\textcolor{defgreen}{\textbf{Def.}}
\begin{itemize}
\item \textbf{Walk:} sequence $v_0, e_1, v_1, \ldots, v_k$ (vertices/edges can repeat)
\item \textbf{Path:} walk with no repeated vertices
\item \textbf{Cycle:} closed walk with no repeated vertices except $v_0 = v_k$
\end{itemize}
\textbf{Length} = number of edges.

\conn{See G-06: Connectivity}\\
\conn{See G-04: Eulerian Graphs}
\tags{\#paths \#cycles}

\tcbitem[zettelstyle, title={\texttt{G-04} --- Eulerian Graphs}]
\footnotesize
\textcolor{defgreen}{\textbf{Def.}}
\begin{itemize}
\item \textbf{Eulerian path:} visits every edge exactly once
\item \textbf{Eulerian circuit:} Eulerian path that starts and ends at same vertex
\end{itemize}

\textcolor{thmred}{\textbf{Thm.}} (Euler, 1736)\\
$G$ has Eulerian circuit $\Leftrightarrow$ $G$ connected and all vertices have even degree.\\
$G$ has Eulerian path $\Leftrightarrow$ exactly 0 or 2 vertices have odd degree.

\conn{See G-02: Degree (Handshaking)}
\tags{\#eulerian \#circuits}

\tcbitem[zettelstyle, title={\texttt{G-05} --- Adjacency Matrix}]
\footnotesize
\textcolor{defgreen}{\textbf{Def.}} For $G$ with $n$ vertices, the \textbf{adjacency matrix} $A \in \{0,1\}^{n \times n}$:
\[
A_{ij} = \begin{cases} 1 & (v_i, v_j) \in E \\ 0 & \text{otherwise} \end{cases}
\]
\textbf{Properties:}
\begin{itemize}
\item Symmetric for undirected graphs
\item $(A^k)_{ij} = $ \# of walks of length $k$ from $v_i$ to $v_j$
\end{itemize}

\conn{See G-01: Graph Definition}\\
\conn{See G-10: Spectral Graph Theory}
\tags{\#representation \#matrix}

\tcbitem[zettelstyle, title={\texttt{G-06} --- Connectivity}]
\footnotesize
\textcolor{defgreen}{\textbf{Def.}}
\begin{itemize}
\item $G$ is \textbf{connected} if $\exists$ path between any two vertices
\item \textbf{Connected component:} maximal connected subgraph
\item \textbf{Cut vertex:} removal disconnects $G$
\item \textbf{Bridge:} edge whose removal disconnects $G$
\end{itemize}

\textcolor{thmred}{\textbf{Thm.}} $G$ connected $\Rightarrow |E| \geq |V| - 1$

\conn{See G-07: Trees}\\
\conn{See G-03: Paths}
\tags{\#connectivity \#components}

\end{tcbitemize}

\newpage

%==========================================================
% PAGE 2: 6 more cards
%==========================================================
\begin{tcbitemize}[
    raster columns=2,
    raster rows=3,
    raster height=0.95\textheight,
    raster equal height=rows,
    raster column skip=4mm,
    raster row skip=4mm,
]

\tcbitem[zettelstyle, title={\texttt{G-07} --- Trees}]
\footnotesize
\textcolor{defgreen}{\textbf{Def.}} A \textbf{tree} is a connected acyclic graph.

\textcolor{thmred}{\textbf{Thm.}} Equivalent definitions for $G$ with $n$ vertices:
\begin{enumerate}
\item $G$ is connected and has $n-1$ edges
\item $G$ is acyclic and has $n-1$ edges
\item $G$ is connected and removing any edge disconnects it
\item Any two vertices connected by unique path
\end{enumerate}

\conn{See G-08: Spanning Trees}\\
\conn{See G-06: Connectivity}
\tags{\#trees \#acyclic}

\tcbitem[zettelstyle, title={\texttt{G-08} --- Spanning Trees}]
\footnotesize
\textcolor{defgreen}{\textbf{Def.}} A \textbf{spanning tree} of $G$ is a subgraph that:
\begin{itemize}
\item Contains all vertices of $G$
\item Is a tree
\end{itemize}

\textcolor{thmred}{\textbf{Thm.}} (Cayley) Number of labeled spanning trees of $K_n$ is $n^{n-2}$.

\textbf{Algorithms:}
\begin{itemize}
\item Kruskal: greedy, sort edges by weight
\item Prim: grow tree from single vertex
\end{itemize}

\conn{See G-07: Trees}
\tags{\#spanning \#mst}

\tcbitem[zettelstyle, title={\texttt{G-09} --- Planar Graphs}]
\footnotesize
\textcolor{defgreen}{\textbf{Def.}} $G$ is \textbf{planar} if it can be drawn in the plane without edge crossings.

\textcolor{thmred}{\textbf{Thm.}} (Euler's Formula) For connected planar graph:
\[
\boxed{V - E + F = 2}
\]
where $F$ = number of faces (including outer).

\textcolor{thmred}{\textbf{Cor.}} For planar $G$ with $|V| \geq 3$: $|E| \leq 3|V| - 6$

\textbf{Non-planar:} $K_5$, $K_{3,3}$

\conn{See G-11: Kuratowski}
\tags{\#planar \#euler}

\tcbitem[zettelstyle, title={\texttt{G-10} --- Graph Coloring}]
\footnotesize
\textcolor{defgreen}{\textbf{Def.}} A \textbf{proper $k$-coloring} assigns colors $\{1, \ldots, k\}$ to vertices s.t. adjacent vertices have different colors.

\textcolor{defgreen}{\textbf{Def.}} \textbf{Chromatic number} $\chi(G)$ = minimum $k$ for proper coloring.

\textbf{Bounds:}
\begin{itemize}
\item $\chi(G) \geq \omega(G)$ (clique number)
\item $\chi(G) \leq \Delta(G) + 1$ (max degree + 1)
\end{itemize}

\textcolor{thmred}{\textbf{Thm.}} (4-Color) Every planar graph is 4-colorable.

\conn{See G-09: Planar Graphs}
\tags{\#coloring \#chromatic}

\tcbitem[zettelstyle, title={\texttt{G-11} --- Bipartite Graphs}]
\footnotesize
\textcolor{defgreen}{\textbf{Def.}} $G = (V, E)$ is \textbf{bipartite} if $V = A \cup B$ with $A \cap B = \emptyset$ and all edges go between $A$ and $B$.

\textcolor{thmred}{\textbf{Thm.}} $G$ is bipartite $\Leftrightarrow$ $G$ has no odd cycles $\Leftrightarrow$ $\chi(G) \leq 2$.

\textbf{Examples:}
\begin{itemize}
\item Trees are bipartite
\item $K_{m,n}$ = complete bipartite
\end{itemize}

\conn{See G-10: Graph Coloring}\\
\conn{See G-12: Matching}
\tags{\#bipartite \#2colorable}

\tcbitem[zettelstyle, title={\texttt{G-12} --- Matching}]
\footnotesize
\textcolor{defgreen}{\textbf{Def.}} A \textbf{matching} $M \subseteq E$ is a set of edges with no shared vertices. \textbf{Perfect matching:} covers all vertices.

\textcolor{thmred}{\textbf{Thm.}} (Hall's Marriage) Bipartite $G = (A \cup B, E)$ has matching covering $A$ $\Leftrightarrow$
\[
|N(S)| \geq |S| \quad \forall S \subseteq A
\]
where $N(S)$ = neighbors of $S$.

\textbf{Algorithm:} Hungarian (poly-time for bipartite).

\conn{See G-11: Bipartite}
\tags{\#matching \#hall}

\end{tcbitemize}

% made with BetterNotes-AI
\begin{flushright}\footnotesize \textit{made with BetterNotes-AI}\end{flushright}

\end{document}

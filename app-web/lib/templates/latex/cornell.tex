\documentclass[10pt]{article}

% =========================
% Preamble (matching existing templates)
% =========================
\usepackage[english]{babel}
\usepackage[utf8]{inputenc}
\usepackage[T1]{fontenc}
\usepackage{enumitem}
\usepackage{amsmath}
\usepackage{amsthm}
\usepackage{amssymb}
\usepackage{amsfonts}
\usepackage{lmodern}
\usepackage{xcolor}
\usepackage{geometry}
\usepackage{fancyhdr}
\usepackage{tikz}
\usepackage{tcolorbox}
\tcbuselibrary{skins, breakable}

% Cornell Layout: Wide left margin for cues
\geometry{a4paper, portrait, left=5.5cm, right=1cm, top=2.5cm, bottom=2cm, marginparwidth=4.5cm, marginparsep=0.5cm}

\setlength{\parindent}{0pt}
\setlength{\parskip}{0.3em}

% Colors
\definecolor{cornellred}{RGB}{179, 27, 27}
\definecolor{cueblue}{RGB}{0, 70, 140}
\definecolor{sectiongreen}{RGB}{0, 120, 60}
\definecolor{defgreen}{RGB}{61,176,0}
\definecolor{propblue}{RGB}{0, 100, 180}
\definecolor{thmred}{RGB}{180, 0, 0}

% Header
\pagestyle{fancy}
\fancyhf{}
\fancyhead[L]{\textbf{\large Linear Algebra --- Cornell Notes}}
\fancyhead[R]{\textbf{Chapter 3: Vector Spaces}}
\renewcommand{\headrulewidth}{1.5pt}
\renewcommand{\headrule}{\hbox to\headwidth{\color{cornellred}\leaders\hrule height \headrulewidth\hfill}}

% Cue command (left margin)
\reversemarginpar
\newcommand{\cue}[1]{%
    \marginpar{\raggedright\footnotesize\textcolor{cueblue}{\textbf{#1}}}%
}

% Theorem-like environments
\newtheoremstyle{cornellstyle}
	{0pt}{3pt}{\upshape}{}{\bfseries}{}{ }{}
\theoremstyle{cornellstyle}
\newtheorem*{definition}{\textcolor{defgreen}{Def.}}
\newtheorem*{proposition}{\textcolor{propblue}{Prop.}}
\newtheorem*{theorem}{\textcolor{thmred}{Thm.}}
\newtheorem*{corollary}{\textcolor{thmred}{Cor.}}
\newtheorem*{example}{\textcolor{sectiongreen}{Ex.}}
\newtheorem*{remark}{\textcolor{cueblue}{Rmk.}}

% Summary box at bottom
\newcommand{\summary}[1]{
\vfill
\begin{tcolorbox}[
    colback=cornellred!5,
    colframe=cornellred,
    title=\textbf{SUMMARY},
    fonttitle=\bfseries\color{white},
    coltitle=white,
    sharp corners,
    boxrule=1pt,
    colbacktitle=cornellred
]
\small #1
\end{tcolorbox}
}

% Section styling
\usepackage{titlesec}
\titleformat{\section}{\color{cornellred}\large\bfseries}{}{0em}{}[\vspace{-0.7em}{\color{cornellred}\rule{\linewidth}{0.5pt}}\vspace{0.3em}]
\titleformat{\subsection}{\color{sectiongreen}\bfseries}{}{0em}{}

\setlist[itemize]{leftmargin=1.2em,itemsep=0.1em,topsep=0.1em}
\setlist[enumerate]{leftmargin=1.2em,itemsep=0.1em,topsep=0.1em}

\begin{document}

%==========================================================
\section{Vector Spaces: Foundations}
%==========================================================

\cue{What is a vector space?}
\begin{definition}
A \textbf{vector space} over a field $\mathbb{F}$ is a set $V$ with two operations:
\begin{enumerate}
\item \textbf{Addition:} $+: V \times V \to V$
\item \textbf{Scalar multiplication:} $\cdot: \mathbb{F} \times V \to V$
\end{enumerate}
satisfying 8 axioms (closure, associativity, commutativity, identities, inverses, distributivity).
\end{definition}

\cue{Key examples?}
\begin{example}
Common vector spaces:
\begin{itemize}
\item $\mathbb{R}^n$ over $\mathbb{R}$ (Euclidean space)
\item $\mathcal{P}_n(\mathbb{R})$ = polynomials of degree $\leq n$
\item $\mathcal{M}_{m \times n}(\mathbb{R})$ = $m \times n$ matrices
\item $C[a,b]$ = continuous functions on $[a,b]$
\end{itemize}
\end{example}

\cue{What is a subspace?}
\begin{definition}
A subset $W \subseteq V$ is a \textbf{subspace} if:
\begin{enumerate}
\item $\mathbf{0} \in W$ (contains zero vector)
\item $\mathbf{u}, \mathbf{v} \in W \Rightarrow \mathbf{u} + \mathbf{v} \in W$ (closed under addition)
\item $c \in \mathbb{F}, \mathbf{v} \in W \Rightarrow c\mathbf{v} \in W$ (closed under scalar mult.)
\end{enumerate}
\end{definition}

\cue{Subspace test shortcut?}
\begin{proposition}
$W \neq \emptyset$ is a subspace $\Leftrightarrow$ $\forall \mathbf{u}, \mathbf{v} \in W, \forall c,d \in \mathbb{F}: c\mathbf{u} + d\mathbf{v} \in W$.
\end{proposition}

%==========================================================
\section{Linear Independence \& Span}
%==========================================================

\cue{Linear combo?}
\begin{definition}
A \textbf{linear combination} of $\mathbf{v}_1, \ldots, \mathbf{v}_n$ is:
\[
c_1\mathbf{v}_1 + c_2\mathbf{v}_2 + \cdots + c_n\mathbf{v}_n, \quad c_i \in \mathbb{F}
\]
\end{definition}

\cue{Span = ?}
\begin{definition}
The \textbf{span} of $\{\mathbf{v}_1, \ldots, \mathbf{v}_n\}$ is the set of all linear combinations:
\[
\text{span}(\mathbf{v}_1, \ldots, \mathbf{v}_n) = \left\{ \sum_{i=1}^n c_i \mathbf{v}_i : c_i \in \mathbb{F} \right\}
\]
This is always a subspace of $V$.
\end{definition}

\cue{Linear independence?}
\begin{definition}
Vectors $\mathbf{v}_1, \ldots, \mathbf{v}_n$ are \textbf{linearly independent} if:
\[
c_1\mathbf{v}_1 + \cdots + c_n\mathbf{v}_n = \mathbf{0} \Rightarrow c_1 = c_2 = \cdots = c_n = 0
\]
Otherwise, they are \textbf{linearly dependent}.
\end{definition}

\cue{How to test?}
\begin{remark}
For $\mathbb{R}^n$: Put vectors as columns in matrix $A$. They are linearly independent $\Leftrightarrow$ $\det(A) \neq 0$ (if square) or $\text{rank}(A) = n$.
\end{remark}

%==========================================================
\section{Basis \& Dimension}
%==========================================================

\cue{What is a basis?}
\begin{definition}
A \textbf{basis} for $V$ is a set $\mathcal{B} = \{\mathbf{v}_1, \ldots, \mathbf{v}_n\}$ such that:
\begin{enumerate}
\item $\mathcal{B}$ is linearly independent
\item $\text{span}(\mathcal{B}) = V$
\end{enumerate}
\end{definition}

\cue{Standard basis?}
\begin{example}
Standard basis for $\mathbb{R}^3$: $\{\mathbf{e}_1, \mathbf{e}_2, \mathbf{e}_3\} = \{(1,0,0), (0,1,0), (0,0,1)\}$

Standard basis for $\mathcal{P}_2(\mathbb{R})$: $\{1, x, x^2\}$
\end{example}

\cue{Dimension?}
\begin{theorem}[Dimension Theorem]
All bases of a finite-dimensional vector space $V$ have the same number of elements. This number is called the \textbf{dimension} of $V$, denoted $\dim(V)$.
\end{theorem}

\cue{Key dimensions?}
\begin{itemize}
\item $\dim(\mathbb{R}^n) = n$
\item $\dim(\mathcal{P}_n(\mathbb{R})) = n + 1$
\item $\dim(\mathcal{M}_{m \times n}) = mn$
\end{itemize}

\cue{Extend/reduce?}
\begin{proposition}
\begin{itemize}
\item Any linearly independent set can be \textbf{extended} to a basis.
\item Any spanning set can be \textbf{reduced} to a basis.
\end{itemize}
\end{proposition}

%==========================================================
\section{Coordinates \& Change of Basis}
%==========================================================

\cue{Coordinate vector?}
\begin{definition}
If $\mathcal{B} = \{\mathbf{v}_1, \ldots, \mathbf{v}_n\}$ is a basis and $\mathbf{x} = c_1\mathbf{v}_1 + \cdots + c_n\mathbf{v}_n$, the \textbf{coordinate vector} of $\mathbf{x}$ w.r.t. $\mathcal{B}$ is:
\[
[\mathbf{x}]_\mathcal{B} = \begin{pmatrix} c_1 \\ \vdots \\ c_n \end{pmatrix} \in \mathbb{F}^n
\]
\end{definition}

\cue{Change of basis?}
\begin{theorem}
Let $\mathcal{B}$ and $\mathcal{C}$ be two bases. The \textbf{change of basis matrix} from $\mathcal{B}$ to $\mathcal{C}$ is $P_{\mathcal{C} \leftarrow \mathcal{B}}$ such that:
\[
[\mathbf{x}]_\mathcal{C} = P_{\mathcal{C} \leftarrow \mathcal{B}} \cdot [\mathbf{x}]_\mathcal{B}
\]
Columns of $P_{\mathcal{C} \leftarrow \mathcal{B}}$ are the $\mathcal{C}$-coordinates of the $\mathcal{B}$-basis vectors.
\end{theorem}

\cue{Inverse?}
\begin{corollary}
$P_{\mathcal{B} \leftarrow \mathcal{C}} = (P_{\mathcal{C} \leftarrow \mathcal{B}})^{-1}$
\end{corollary}

%==========================================================
\section{Rank-Nullity Theorem}
%==========================================================

\cue{Kernel/Image?}
\begin{definition}
For linear map $T: V \to W$:
\begin{itemize}
\item \textbf{Kernel:} $\ker(T) = \{\mathbf{v} \in V : T(\mathbf{v}) = \mathbf{0}\}$ (nullspace)
\item \textbf{Image:} $\text{Im}(T) = \{T(\mathbf{v}) : \mathbf{v} \in V\}$ (range)
\end{itemize}
\end{definition}

\cue{Rank-Nullity?}
\begin{theorem}[Rank-Nullity]
For $T: V \to W$ linear with $\dim(V) < \infty$:
\[
\boxed{\dim(\ker T) + \dim(\text{Im } T) = \dim(V)}
\]
Equivalently: $\text{nullity}(T) + \text{rank}(T) = \dim(V)$
\end{theorem}

\cue{For matrices?}
\begin{remark}
For $A \in \mathcal{M}_{m \times n}$: $\text{nullity}(A) + \text{rank}(A) = n$ (number of columns).
\end{remark}

%==========================================================
\summary{
\textbf{Vector space:} Set with addition \& scalar mult. satisfying 8 axioms. \textbf{Subspace:} Contains $\mathbf{0}$, closed under $+$ and $\cdot$. \textbf{Span:} All linear combos. \textbf{Linearly independent:} Only trivial solution to $\sum c_i\mathbf{v}_i = 0$. \textbf{Basis:} Linearly indep. + spans $V$. \textbf{Dimension:} \# of basis vectors. \textbf{Rank-Nullity:} $\dim(\ker T) + \dim(\text{Im } T) = \dim(V)$.
}

% made with BetterNotes-AI
\begin{flushright}\footnotesize \textit{made with BetterNotes-AI}\end{flushright}

\end{document}

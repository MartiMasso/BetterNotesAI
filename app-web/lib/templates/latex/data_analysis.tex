\documentclass[11pt]{article}

% =============================================================================
% DATA ANALYSIS REPORT TEMPLATE - PRO
% For: Bioinformaticians, Data Scientists, Statisticians, ML Researchers
% Style: Technical report with statistical rigor
% =============================================================================

\usepackage[english]{babel}
\usepackage[utf8]{inputenc}
\usepackage[T1]{fontenc}
\usepackage{lmodern}
\usepackage{microtype}

% Page geometry
\usepackage{geometry}
\geometry{a4paper, left=25mm, right=25mm, top=25mm, bottom=25mm}

% Mathematics
\usepackage{amsmath}
\usepackage{amssymb}
\usepackage{bm}

% Graphics and tables
\usepackage{graphicx}
\usepackage{float}
\usepackage{booktabs}
\usepackage{array}
\usepackage{multirow}
\usepackage{tabularx}
\usepackage{longtable}
\usepackage{caption}
\captionsetup{font=small, labelfont=bf}

% Colors
\usepackage{xcolor}
\definecolor{codebg}{RGB}{248,248,248}
\definecolor{codeframe}{RGB}{200,200,200}
\definecolor{significant}{RGB}{180,0,0}
\definecolor{headerblue}{RGB}{0,51,102}

% Code listings
\usepackage{listings}
\lstdefinestyle{python}{
    language=Python,
    basicstyle=\ttfamily\small,
    keywordstyle=\color{blue!70!black},
    stringstyle=\color{red!60!black},
    commentstyle=\color{green!50!black}\itshape,
    backgroundcolor=\color{codebg},
    frame=single,
    framerule=0.5pt,
    rulecolor=\color{codeframe},
    numbers=left,
    numberstyle=\tiny\color{gray},
    breaklines=true,
    showstringspaces=false,
    tabsize=4,
    xleftmargin=2em,
    framexleftmargin=1.5em
}
\lstdefinestyle{R}{
    language=R,
    basicstyle=\ttfamily\small,
    keywordstyle=\color{blue!70!black},
    stringstyle=\color{red!60!black},
    commentstyle=\color{green!50!black}\itshape,
    backgroundcolor=\color{codebg},
    frame=single,
    framerule=0.5pt,
    rulecolor=\color{codeframe},
    numbers=left,
    numberstyle=\tiny\color{gray},
    breaklines=true,
    showstringspaces=false,
    tabsize=4,
    xleftmargin=2em,
    framexleftmargin=1.5em
}

% Links
\usepackage{hyperref}
\hypersetup{colorlinks=true, linkcolor=headerblue, citecolor=headerblue, urlcolor=blue}

% Section formatting
\usepackage{titlesec}
\titleformat{\section}{\large\bfseries\color{headerblue}}{\thesection.}{0.5em}{}
\titleformat{\subsection}{\normalsize\bfseries}{\thesubsection}{0.5em}{}
\titleformat{\subsubsection}{\normalsize\itshape}{\thesubsubsection}{0.5em}{}

% Header/footer
\usepackage{fancyhdr}
\pagestyle{fancy}
\fancyhf{}
\fancyhead[L]{\small\textit{Gene Expression Analysis}}
\fancyhead[R]{\small\textit{Data Analysis Report}}
\fancyfoot[C]{\thepage}
\renewcommand{\headrulewidth}{0.4pt}

% Custom environments
\usepackage{tcolorbox}
\tcbuselibrary{breakable,skins}

% Dataset summary box
\newtcolorbox{datasummary}[1][]{
    colback=blue!5,
    colframe=blue!50!black,
    fonttitle=\bfseries,
    title=Dataset Summary,
    #1
}

% Key finding box
\newtcolorbox{keyfinding}[1][]{
    colback=green!5,
    colframe=green!50!black,
    fonttitle=\bfseries,
    title=Key Finding,
    #1
}

% Warning/limitation box
\newtcolorbox{limitation}[1][]{
    colback=orange!5,
    colframe=orange!50!black,
    fonttitle=\bfseries,
    title=Limitation,
    #1
}

% Statistical significance notation
\newcommand{\pval}[1]{\textit{p} = #1}
\newcommand{\pless}[1]{\textit{p} < #1}
\newcommand{\signif}{\textcolor{significant}{*}}
\newcommand{\signifb}{\textcolor{significant}{**}}
\newcommand{\signifc}{\textcolor{significant}{***}}

% Effect size notation
\newcommand{\cohend}{Cohen's \textit{d}}
\newcommand{\etasq}{\eta^2}
\newcommand{\rsq}{R^2}

\begin{document}

% =============================================================================
% TITLE PAGE
% =============================================================================
\begin{titlepage}
\centering

\vspace*{1.5cm}

{\Large\textsc{Computational Biology Group}}\\[0.3cm]
{\large\textsc{Institute for Biomedical Research}}\\[2cm]

\rule{\linewidth}{0.5pt}\\[0.4cm]
{\LARGE\bfseries Differential Gene Expression Analysis\\[0.2cm] in Breast Cancer Subtypes}\\[0.3cm]
\rule{\linewidth}{0.5pt}\\[1.5cm]

{\large\textbf{Technical Report}}\\[0.3cm]
{\large Project ID: TCGA-BRCA-2024-001}\\[1cm]

\vfill

\begin{tabular}{rl}
\textbf{Lead Analyst:} & Dr. Jane Doe \\[0.2cm]
\textbf{Collaborators:} & Dr. John Smith, Dr. Alice Johnson \\[0.2cm]
\textbf{Analysis Date:} & January 15, 2024 \\[0.2cm]
\textbf{Data Source:} & TCGA (The Cancer Genome Atlas) \\[0.2cm]
\textbf{Pipeline Version:} & v2.3.1 \\
\end{tabular}

\vfill

{\small
\textbf{Software:} R 4.3.2, DESeq2 1.40.0, clusterProfiler 4.8.0\\
\textbf{Repository:} \url{https://github.com/example/brca-analysis}
}

\end{titlepage}

% =============================================================================
% EXECUTIVE SUMMARY
% =============================================================================
\section*{Executive Summary}
\addcontentsline{toc}{section}{Executive Summary}

This report presents a comprehensive differential gene expression analysis comparing four molecular subtypes of breast cancer (Luminal A, Luminal B, HER2-enriched, and Basal-like) using RNA-seq data from the TCGA-BRCA cohort.

\begin{keyfinding}
\begin{itemize}
    \item \textbf{2,847 differentially expressed genes} identified (adj. $p < 0.01$, $|\log_2\text{FC}| > 1$)
    \item \textbf{ESR1} and \textbf{ERBB2} expression patterns strongly associate with clinical subtypes ($\rsq = 0.78$)
    \item \textbf{Basal-like tumors} show enrichment for cell cycle and DNA repair pathways (FDR $< 10^{-15}$)
    \item \textbf{Machine learning classifier} achieves 94.2\% accuracy (5-fold CV) distinguishing subtypes
\end{itemize}
\end{keyfinding}

\tableofcontents
\newpage

% =============================================================================
% 1. INTRODUCTION
% =============================================================================
\section{Introduction}

\subsection{Background}

Breast cancer is a heterogeneous disease with distinct molecular subtypes defined by gene expression profiles~\cite{perou2000,sorlie2001}. The PAM50 classifier identifies four intrinsic subtypes based on 50 genes:

\begin{enumerate}
    \item \textbf{Luminal A}: ER+/HER2-, low proliferation, best prognosis
    \item \textbf{Luminal B}: ER+/HER2- or ER+/HER2+, high proliferation
    \item \textbf{HER2-enriched}: HER2+, ER-/PR-
    \item \textbf{Basal-like}: ER-/PR-/HER2-, worst prognosis
\end{enumerate}

Understanding the transcriptomic differences between subtypes is essential for developing targeted therapies and prognostic biomarkers.

\subsection{Objectives}

\begin{enumerate}
    \item Identify differentially expressed genes (DEGs) between breast cancer subtypes
    \item Perform pathway enrichment analysis to characterize biological processes
    \item Develop a predictive model for subtype classification
    \item Validate findings against independent datasets
\end{enumerate}

\subsection{Data Source}

\begin{datasummary}
\begin{tabular}{@{}ll@{}}
\textbf{Project:} & TCGA-BRCA (Breast Invasive Carcinoma) \\
\textbf{Samples:} & $n = 1,097$ primary tumors \\
\textbf{Platform:} & Illumina HiSeq RNA-seq \\
\textbf{Normalization:} & FPKM $\rightarrow$ TPM \\
\textbf{Genes:} & 20,531 protein-coding genes \\
\textbf{Clinical data:} & Age, stage, subtype, survival \\
\end{tabular}
\end{datasummary}

% =============================================================================
% 2. METHODS
% =============================================================================
\section{Methods}

\subsection{Data Preprocessing}

Raw RNA-seq counts were obtained from the GDC Data Portal. Preprocessing steps included:

\begin{lstlisting}[style=R, caption={Data preprocessing pipeline in R}]
library(DESeq2)
library(edgeR)

# Load count matrix and metadata
counts <- read.csv("TCGA_BRCA_counts.csv", row.names=1)
metadata <- read.csv("TCGA_BRCA_clinical.csv")

# Filter low-expression genes (CPM > 1 in at least 10% of samples)
keep <- rowSums(cpm(counts) > 1) >= 0.1 * ncol(counts)
counts_filtered <- counts[keep, ]
# Retained: 15,847 genes

# Create DESeq2 object
dds <- DESeqDataSetFromMatrix(
    countData = counts_filtered,
    colData = metadata,
    design = ~ subtype
)

# Variance stabilizing transformation for visualization
vsd <- vst(dds, blind = FALSE)
\end{lstlisting}

\subsection{Differential Expression Analysis}

Pairwise comparisons between subtypes were performed using DESeq2 with the following thresholds:
\begin{itemize}
    \item Adjusted $p$-value (Benjamini-Hochberg FDR) $< 0.01$
    \item $|\log_2\text{fold change}| > 1.0$
\end{itemize}

The statistical model accounts for the negative binomial distribution of count data:
\begin{equation}
    K_{ij} \sim \text{NB}(\mu_{ij}, \alpha_i),
\end{equation}
where $K_{ij}$ is the count for gene $i$ in sample $j$, $\mu_{ij}$ is the expected value, and $\alpha_i$ is the dispersion parameter.

\subsection{Pathway Enrichment Analysis}

Gene Ontology (GO) and KEGG pathway enrichment were performed using clusterProfiler~\cite{yu2012}:

\begin{lstlisting}[style=R, caption={Pathway enrichment analysis}]
library(clusterProfiler)
library(org.Hs.eg.db)

# Convert gene symbols to Entrez IDs
gene_list <- bitr(DEGs$gene_symbol, 
                  fromType = "SYMBOL",
                  toType = "ENTREZID",
                  OrgDb = org.Hs.eg.db)

# GO Biological Process enrichment
go_bp <- enrichGO(gene = gene_list$ENTREZID,
                  OrgDb = org.Hs.eg.db,
                  ont = "BP",
                  pAdjustMethod = "BH",
                  pvalueCutoff = 0.05,
                  qvalueCutoff = 0.05)

# KEGG pathway enrichment
kegg <- enrichKEGG(gene = gene_list$ENTREZID,
                   organism = 'hsa',
                   pvalueCutoff = 0.05)
\end{lstlisting}

\subsection{Machine Learning Classification}

A random forest classifier was trained to predict subtype from gene expression:
\begin{itemize}
    \item \textbf{Features:} Top 500 DEGs (by variance)
    \item \textbf{Model:} Random Forest (500 trees, mtry = 22)
    \item \textbf{Validation:} Stratified 5-fold cross-validation
    \item \textbf{Metrics:} Accuracy, precision, recall, F1-score
\end{itemize}

% =============================================================================
% 3. RESULTS
% =============================================================================
\section{Results}

\subsection{Sample Distribution}

Table~\ref{tab:sample_distribution} shows the distribution of samples across subtypes and clinical stages.

\begin{table}[htbp]
\centering
\caption{Sample distribution by molecular subtype and tumor stage.}
\label{tab:sample_distribution}
\begin{tabular}{@{}lrrrrr@{}}
\toprule
& \multicolumn{4}{c}{\textbf{Stage}} & \\
\cmidrule(lr){2-5}
\textbf{Subtype} & I & II & III & IV & \textbf{Total} \\
\midrule
Luminal A & 98 & 187 & 72 & 8 & 365 (33.3\%) \\
Luminal B & 61 & 124 & 58 & 5 & 248 (22.6\%) \\
HER2-enriched & 29 & 68 & 41 & 3 & 141 (12.9\%) \\
Basal-like & 37 & 89 & 55 & 7 & 188 (17.1\%) \\
Normal-like & 23 & 41 & 27 & 4 & 95 (8.7\%) \\
Unclassified & 18 & 29 & 11 & 2 & 60 (5.5\%) \\
\midrule
\textbf{Total} & 266 & 538 & 264 & 29 & \textbf{1,097} \\
\bottomrule
\end{tabular}
\end{table}

\subsection{Differential Expression Results}

A total of \textbf{2,847 differentially expressed genes} were identified in at least one pairwise comparison (adj. $p < 0.01$, $|\log_2\text{FC}| > 1$).

\begin{table}[htbp]
\centering
\caption{Number of differentially expressed genes in pairwise subtype comparisons.}
\label{tab:deg_counts}
\begin{tabular}{@{}lccc@{}}
\toprule
\textbf{Comparison} & \textbf{Up} & \textbf{Down} & \textbf{Total} \\
\midrule
Basal vs. Luminal A & 1,284 & 967 & 2,251 \\
Basal vs. Luminal B & 1,031 & 742 & 1,773 \\
Basal vs. HER2 & 687 & 412 & 1,099 \\
HER2 vs. Luminal A & 534 & 398 & 932 \\
HER2 vs. Luminal B & 321 & 287 & 608 \\
Luminal B vs. Luminal A & 187 & 154 & 341 \\
\bottomrule
\end{tabular}
\end{table}

\subsection{Top Differentially Expressed Genes}

Table~\ref{tab:top_genes} lists the top 15 genes ranked by adjusted $p$-value across all comparisons.

\begin{table}[htbp]
\centering
\caption{Top 15 differentially expressed genes (Basal vs. Luminal A comparison).}
\label{tab:top_genes}
\begin{tabular}{@{}llrrrl@{}}
\toprule
\textbf{Gene} & \textbf{Description} & \textbf{log$_2$FC} & \textbf{SE} & \textbf{adj. \textit{p}} & \textbf{Sig.} \\
\midrule
ESR1 & Estrogen receptor 1 & $-4.82$ & 0.18 & $<10^{-100}$ & \signifc \\
GATA3 & GATA binding protein 3 & $-3.94$ & 0.15 & $<10^{-95}$ & \signifc \\
FOXA1 & Forkhead box A1 & $-3.67$ & 0.14 & $<10^{-87}$ & \signifc \\
KRT5 & Keratin 5 & $+4.21$ & 0.19 & $<10^{-78}$ & \signifc \\
KRT14 & Keratin 14 & $+3.89$ & 0.17 & $<10^{-72}$ & \signifc \\
FOXC1 & Forkhead box C1 & $+3.54$ & 0.16 & $<10^{-65}$ & \signifc \\
MKI67 & Marker of proliferation Ki-67 & $+2.87$ & 0.12 & $<10^{-58}$ & \signifc \\
ERBB2 & HER2/neu receptor & $+0.42$ & 0.08 & $3.2 \times 10^{-7}$ & \signifc \\
PGR & Progesterone receptor & $-3.21$ & 0.14 & $<10^{-55}$ & \signifc \\
CDH1 & E-cadherin & $-1.54$ & 0.09 & $<10^{-42}$ & \signifc \\
CCNB1 & Cyclin B1 & $+2.34$ & 0.11 & $<10^{-40}$ & \signifc \\
TOP2A & Topoisomerase II alpha & $+2.67$ & 0.12 & $<10^{-38}$ & \signifc \\
BRCA1 & BRCA1 DNA repair & $+1.12$ & 0.07 & $<10^{-28}$ & \signifc \\
TP53 & Tumor protein p53 & $+0.87$ & 0.06 & $<10^{-22}$ & \signifc \\
MYC & MYC proto-oncogene & $+1.34$ & 0.08 & $<10^{-19}$ & \signifc \\
\bottomrule
\end{tabular}
\begin{flushleft}
\small\signifc~adj. $p < 0.001$; \signifb~adj. $p < 0.01$; \signif~adj. $p < 0.05$
\end{flushleft}
\end{table}

\subsection{Pathway Enrichment}

\subsubsection{Basal-like Enriched Pathways}

Genes upregulated in Basal-like tumors were significantly enriched in:

\begin{table}[htbp]
\centering
\caption{Top GO Biological Process terms enriched in Basal-like tumors.}
\label{tab:go_basal}
\begin{tabular}{@{}lcccr@{}}
\toprule
\textbf{GO Term} & \textbf{Genes} & \textbf{Fold Enrich.} & \textbf{FDR} \\
\midrule
Cell cycle (GO:0007049) & 187 & 3.4 & $<10^{-45}$ \\
DNA replication (GO:0006260) & 89 & 4.2 & $<10^{-32}$ \\
Mitotic nuclear division (GO:0140014) & 124 & 3.8 & $<10^{-28}$ \\
DNA repair (GO:0006281) & 112 & 2.9 & $<10^{-21}$ \\
Chromosome segregation (GO:0007059) & 67 & 4.1 & $<10^{-18}$ \\
\bottomrule
\end{tabular}
\end{table}

\subsubsection{KEGG Pathway Analysis}

\begin{table}[htbp]
\centering
\caption{Significantly enriched KEGG pathways.}
\label{tab:kegg}
\begin{tabular}{@{}lllcr@{}}
\toprule
\textbf{Pathway ID} & \textbf{Pathway Name} & \textbf{Category} & \textbf{Genes} & \textbf{FDR} \\
\midrule
hsa04110 & Cell cycle & Cell growth & 78 & $<10^{-25}$ \\
hsa03030 & DNA replication & Replication & 34 & $<10^{-18}$ \\
hsa04115 & p53 signaling pathway & Signal trans. & 45 & $<10^{-12}$ \\
hsa03440 & Homologous recombination & Repair & 28 & $<10^{-10}$ \\
hsa04512 & ECM-receptor interaction & Adhesion & 52 & $<10^{-8}$ \\
\bottomrule
\end{tabular}
\end{table}

\subsection{Classification Performance}

The random forest classifier achieved excellent performance in distinguishing subtypes:

\begin{table}[htbp]
\centering
\caption{Classification performance metrics (5-fold cross-validation).}
\label{tab:ml_performance}
\begin{tabular}{@{}lcccc@{}}
\toprule
\textbf{Subtype} & \textbf{Precision} & \textbf{Recall} & \textbf{F1-Score} & \textbf{Support} \\
\midrule
Luminal A & 0.96 & 0.94 & 0.95 & 365 \\
Luminal B & 0.89 & 0.91 & 0.90 & 248 \\
HER2-enriched & 0.92 & 0.88 & 0.90 & 141 \\
Basal-like & 0.98 & 0.99 & 0.98 & 188 \\
\midrule
\textbf{Macro avg} & 0.94 & 0.93 & 0.93 & 942 \\
\textbf{Weighted avg} & 0.94 & 0.94 & 0.94 & 942 \\
\midrule
\multicolumn{4}{l}{\textbf{Overall Accuracy:}} & \textbf{94.2\%} \\
\bottomrule
\end{tabular}
\end{table}

The top 10 most important features (by Gini importance) were: ESR1, GATA3, FOXA1, KRT5, KRT14, MKI67, FOXC1, PGR, CDH1, and ERBB2.

% =============================================================================
% 4. DISCUSSION
% =============================================================================
\section{Discussion}

\subsection{Biological Interpretation}

The differential expression results are consistent with known biology of breast cancer subtypes:

\begin{itemize}
    \item \textbf{Luminal tumors:} High expression of ESR1, GATA3, FOXA1 reflects estrogen receptor signaling dependency.
    
    \item \textbf{Basal-like tumors:} Upregulation of basal keratins (KRT5, KRT14) and proliferation markers (MKI67) aligns with their aggressive phenotype.
    
    \item \textbf{HER2-enriched:} ERBB2 amplification drives the distinct expression pattern.
\end{itemize}

The enrichment of cell cycle and DNA repair pathways in Basal-like tumors explains their sensitivity to platinum chemotherapy and PARP inhibitors.

\subsection{Clinical Implications}

\begin{keyfinding}
The gene signature identified here could serve as:
\begin{enumerate}
    \item A diagnostic tool for subtype classification
    \item Prognostic markers for survival prediction
    \item Targets for subtype-specific therapies
\end{enumerate}
\end{keyfinding}

\subsection{Limitations}

\begin{limitation}
\begin{itemize}
    \item \textbf{Batch effects:} TCGA data was collected across multiple centers; while we applied batch correction, residual effects may persist.
    
    \item \textbf{Tumor heterogeneity:} Bulk RNA-seq averages expression across cell types; single-cell analysis would provide higher resolution.
    
    \item \textbf{External validation:} Results should be validated in independent cohorts (e.g., METABRIC, GSE96058).
\end{itemize}
\end{limitation}

% =============================================================================
% 5. CONCLUSIONS
% =============================================================================
\section{Conclusions}

\begin{enumerate}
    \item We identified 2,847 differentially expressed genes distinguishing breast cancer subtypes.
    
    \item Basal-like tumors show strong enrichment for cell cycle and DNA repair pathways.
    
    \item A machine learning classifier achieves 94.2\% accuracy in subtype prediction.
    
    \item ESR1, GATA3, and FOXA1 are the strongest discriminators of luminal vs. non-luminal subtypes.
\end{enumerate}

\subsection{Future Directions}

\begin{itemize}
    \item Integration with proteomics and metabolomics data
    \item Single-cell RNA-seq analysis for tumor microenvironment characterization
    \item Survival analysis incorporating the identified gene signature
\end{itemize}

% =============================================================================
% REFERENCES
% =============================================================================
\begin{thebibliography}{9}

\bibitem{perou2000}
C. M. Perou et al., ``Molecular portraits of human breast tumours,'' \textit{Nature} \textbf{406}, 747--752 (2000).

\bibitem{sorlie2001}
T. Sørlie et al., ``Gene expression patterns of breast carcinomas distinguish tumor subclasses with clinical implications,'' \textit{PNAS} \textbf{98}, 10869--10874 (2001).

\bibitem{love2014}
M. I. Love, W. Huber, and S. Anders, ``Moderated estimation of fold change and dispersion for RNA-seq data with DESeq2,'' \textit{Genome Biology} \textbf{15}, 550 (2014).

\bibitem{yu2012}
G. Yu et al., ``clusterProfiler: an R package for comparing biological themes among gene clusters,'' \textit{OMICS} \textbf{16}, 284--287 (2012).

\bibitem{koboldt2012}
D. C. Koboldt et al., ``Comprehensive molecular portraits of human breast tumours,'' \textit{Nature} \textbf{490}, 61--70 (2012) [TCGA].

\end{thebibliography}

% =============================================================================
% APPENDIX
% =============================================================================
\appendix
\section{Supplementary Tables}

Full gene lists and pathway enrichment results are available at:
\begin{center}
\url{https://github.com/example/brca-analysis/supplementary}
\end{center}

\section{Session Information}

\begin{lstlisting}[style=R, caption={R session info for reproducibility}]
R version 4.3.2 (2023-10-31)
Platform: x86_64-pc-linux-gnu (64-bit)

attached packages:
- DESeq2_1.40.0
- clusterProfiler_4.8.0  
- org.Hs.eg.db_3.17.0
- ggplot2_3.4.4
- dplyr_1.1.4
- randomForest_4.7-1.1
\end{lstlisting}

\vfill
\begin{flushright}
\footnotesize\textit{made with BetterNotes-AI}
\end{flushright}

\end{document}

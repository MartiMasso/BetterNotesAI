\documentclass[11pt]{article}

% =============================================================================
% TECHNICAL LAB REPORT TEMPLATE - PRO
% For: Engineers, Experimental Physicists, Chemistry/Materials Science
% Style: Professional technical report with error analysis
% =============================================================================

\usepackage[english]{babel}
\usepackage[utf8]{inputenc}
\usepackage[T1]{fontenc}
\usepackage{lmodern}
\usepackage{microtype}

% Page geometry
\usepackage{geometry}
\geometry{a4paper, left=25mm, right=25mm, top=25mm, bottom=30mm}

% Mathematics
\usepackage{amsmath}
\usepackage{amssymb}
\usepackage{siunitx}
\sisetup{
    separate-uncertainty=true,
    multi-part-units=single,
    per-mode=symbol
}

% Graphics and tables
\usepackage{graphicx}
\usepackage{float}
\usepackage{booktabs}
\usepackage{array}
\usepackage{multirow}
\usepackage{tabularx}
\usepackage{caption}
\captionsetup{font=small, labelfont=bf}

% Code listings (for equipment/data)
\usepackage{listings}
\lstset{
    basicstyle=\ttfamily\small,
    breaklines=true,
    frame=single,
    numbers=left,
    numberstyle=\tiny,
    backgroundcolor=\color{gray!10}
}

% Colors
\usepackage{xcolor}
\definecolor{darkblue}{RGB}{0,51,102}

% Links
\usepackage{hyperref}
\hypersetup{colorlinks=true, linkcolor=darkblue, citecolor=darkblue, urlcolor=blue}

% Section formatting
\usepackage{titlesec}
\titleformat{\section}{\large\bfseries}{\thesection.}{0.5em}{}
\titleformat{\subsection}{\normalsize\bfseries}{\thesubsection}{0.5em}{}
\titleformat{\subsubsection}{\normalsize\itshape}{\thesubsubsection}{0.5em}{}

% Header/footer
\usepackage{fancyhdr}
\pagestyle{fancy}
\fancyhf{}
\fancyhead[L]{\small\textit{PHY4501 -- Advanced Laboratory}}
\fancyhead[R]{\small\textit{Photoelectric Effect}}
\fancyfoot[C]{\thepage}
\renewcommand{\headrulewidth}{0.4pt}

% Custom commands for uncertainties
\newcommand{\uncertainty}[2]{#1 \pm #2}
\newcommand{\result}[3]{#1 = \SI{#2 \pm #3}}
\newcommand{\measured}[2]{\SI{#1}{#2}}

% Boxed equations for final results
\usepackage{empheq}
\newcommand{\boxedeq}[1]{\begin{empheq}[box=\fbox]{equation}#1\end{empheq}}

\begin{document}

% =============================================================================
% TITLE PAGE
% =============================================================================
\begin{titlepage}
\centering

\vspace*{2cm}

{\Large\textsc{Department of Physics}}\\[0.3cm]
{\large\textsc{University of Example}}\\[2cm]

\rule{\linewidth}{0.5pt}\\[0.4cm]
{\LARGE\bfseries Determination of Planck's Constant\\[0.2cm] via the Photoelectric Effect}\\[0.3cm]
\rule{\linewidth}{0.5pt}\\[1.5cm]

{\large\textbf{PHY4501 -- Advanced Laboratory Physics}}\\[0.5cm]

\vfill

\begin{tabular}{rl}
\textbf{Author:} & Student Name (ID: 12345678) \\[0.2cm]
\textbf{Lab Partner:} & Partner Name \\[0.2cm]
\textbf{Date of Experiment:} & 15 January 2024 \\[0.2cm]
\textbf{Date Submitted:} & 22 January 2024 \\[0.2cm]
\textbf{Demonstrator:} & Dr. J. Smith \\
\end{tabular}

\vfill

{\small Word count: 2,847}

\end{titlepage}

% =============================================================================
% ABSTRACT
% =============================================================================
\begin{abstract}
\noindent
The photoelectric effect was used to determine Planck's constant by measuring the stopping potential as a function of incident light frequency. A mercury lamp with interference filters provided five discrete wavelengths (\SIrange{365}{578}{\nano\meter}). Linear regression of stopping potential versus frequency yielded $h = \SI{6.58 \pm 0.12 e-34}{\joule\second}$, in agreement with the accepted value of $h = \SI{6.626 e-34}{\joule\second}$ within experimental uncertainty. The work function of the potassium cathode was determined to be $\phi = \SI{2.14 \pm 0.08}{\electronvolt}$. Systematic errors from contact potential differences and stray light were analyzed. The relative uncertainty of $1.8\%$ demonstrates the precision achievable with careful technique.
\end{abstract}

\tableofcontents
\newpage

% =============================================================================
% 1. INTRODUCTION
% =============================================================================
\section{Introduction}

\subsection{Theoretical Background}

The photoelectric effect---the emission of electrons from a metal surface when illuminated by light---was first observed by Hertz in 1887 and explained by Einstein in 1905~\cite{einstein1905}. Einstein proposed that light consists of discrete quanta (photons) with energy
\begin{equation}
    E_{\text{photon}} = h\nu,
    \label{eq:photon_energy}
\end{equation}
where $h$ is Planck's constant and $\nu$ is the frequency of light.

When a photon is absorbed by an electron in the metal, the electron gains sufficient energy to overcome the work function $\phi$ (the minimum energy required to escape the surface) and be ejected with kinetic energy:
\begin{equation}
    K_{\text{max}} = h\nu - \phi.
    \label{eq:photoelectric}
\end{equation}

The stopping potential $V_s$ is the retarding voltage required to halt the most energetic photoelectrons:
\begin{equation}
    eV_s = K_{\text{max}} = h\nu - \phi,
    \label{eq:stopping}
\end{equation}
where $e$ is the elementary charge. Rearranging:
\begin{equation}
    V_s = \frac{h}{e}\nu - \frac{\phi}{e}.
    \label{eq:linear}
\end{equation}

This equation predicts a linear relationship between $V_s$ and $\nu$, with slope $h/e$ and intercept $-\phi/e$.

\subsection{Objectives}

The primary objectives of this experiment are:
\begin{enumerate}
    \item Verify the linear relationship between stopping potential and frequency predicted by Eq.~\eqref{eq:linear}.
    \item Determine Planck's constant $h$ from the slope of the $V_s$--$\nu$ relationship.
    \item Calculate the work function $\phi$ of the potassium photocathode.
    \item Analyze systematic and random uncertainties.
\end{enumerate}

% =============================================================================
% 2. EXPERIMENTAL METHOD
% =============================================================================
\section{Experimental Method}

\subsection{Apparatus}

The experimental setup consisted of:
\begin{itemize}
    \item PASCO h/e Apparatus (Model AP-9370)
    \item Mercury discharge lamp (Hg spectral source, 100W)
    \item Interference filters: \SI{365}{\nano\meter}, \SI{405}{\nano\meter}, \SI{436}{\nano\meter}, \SI{546}{\nano\meter}, \SI{578}{\nano\meter}
    \item Digital voltmeter (Keithley 2000, resolution \SI{0.1}{\milli\volt})
    \item Optical rail and mounts
    \item Light-tight enclosure
\end{itemize}

The h/e apparatus contains a vacuum phototube with potassium photocathode and collecting anode. Built-in electronics allow direct measurement of stopping potential.

\subsection{Procedure}

\begin{enumerate}
    \item The apparatus was allowed to warm up for 30 minutes to achieve thermal equilibrium.
    
    \item The mercury lamp was positioned \SI{40}{\centi\meter} from the phototube entrance aperture.
    
    \item For each wavelength:
    \begin{enumerate}
        \item The appropriate interference filter was inserted.
        \item The voltage display was zeroed with the shutter closed.
        \item The shutter was opened and the stopping potential recorded after stabilization ($\sim$\SI{30}{\second}).
        \item Three independent measurements were taken.
    \end{enumerate}
    
    \item Background measurements with an opaque filter confirmed zero current in darkness.
    
    \item Room lights were extinguished to minimize stray light.
\end{enumerate}

% =============================================================================
% 3. DATA AND ANALYSIS
% =============================================================================
\section{Data and Analysis}

\subsection{Raw Data}

Table~\ref{tab:raw_data} presents the measured stopping potentials for each wavelength. The frequency was calculated from $\nu = c/\lambda$ using $c = \SI{2.998e8}{\meter\per\second}$.

\begin{table}[htbp]
\centering
\caption{Measured stopping potentials for five mercury spectral lines.}
\label{tab:raw_data}
\begin{tabular}{@{}cccccc@{}}
\toprule
$\lambda$ & $\nu$ & $V_{s,1}$ & $V_{s,2}$ & $V_{s,3}$ & $\bar{V}_s$ \\
(\si{\nano\meter}) & (\SI{e14}{\hertz}) & (\si{\volt}) & (\si{\volt}) & (\si{\volt}) & (\si{\volt}) \\
\midrule
365.0 & 8.214 & 1.721 & 1.718 & 1.724 & $1.721 \pm 0.003$ \\
404.7 & 7.408 & 1.402 & 1.398 & 1.405 & $1.402 \pm 0.004$ \\
435.8 & 6.879 & 1.181 & 1.177 & 1.183 & $1.180 \pm 0.003$ \\
546.1 & 5.490 & 0.627 & 0.631 & 0.625 & $0.628 \pm 0.003$ \\
577.0 & 5.196 & 0.508 & 0.512 & 0.505 & $0.508 \pm 0.004$ \\
\bottomrule
\end{tabular}
\end{table}

The uncertainty in the mean was calculated using the standard error:
\begin{equation}
    \sigma_{\bar{V}} = \frac{\sigma}{\sqrt{n}} = \frac{\sigma}{\sqrt{3}},
\end{equation}
where $\sigma$ is the standard deviation of the three measurements.

\subsection{Linear Regression Analysis}

The data were fitted to the linear model $V_s = m\nu + c$ using weighted least-squares regression. The weights were taken as $w_i = 1/\sigma_i^2$.

The weighted least-squares estimators are:
\begin{align}
    m &= \frac{S_{xy}}{S_{xx}} = \frac{\sum w_i(\nu_i - \bar{\nu})(V_{s,i} - \bar{V}_s)}{\sum w_i(\nu_i - \bar{\nu})^2}, \\
    c &= \bar{V}_s - m\bar{\nu},
\end{align}
where overbars denote weighted means.

The resulting fit parameters are:
\begin{align}
    m &= \SI{4.109 \pm 0.074 e-15}{\volt\per\hertz}, \\
    c &= \SI{-2.143 \pm 0.052}{\volt}.
\end{align}

The coefficient of determination $R^2 = 0.9987$ indicates excellent linearity.

\subsection{Determination of Planck's Constant}

From Eq.~\eqref{eq:linear}, the slope equals $h/e$. Therefore:
\begin{equation}
    h = m \cdot e = \SI{4.109 e-15}{\volt\per\hertz} \times \SI{1.602 e-19}{\coulomb}.
\end{equation}

\boxedeq{h = \SI{6.58 \pm 0.12 e-34}{\joule\second}}

The uncertainty in $h$ was propagated from the slope uncertainty:
\begin{equation}
    \frac{\sigma_h}{h} = \frac{\sigma_m}{m} = \frac{0.074}{4.109} = 1.8\%.
\end{equation}

\subsection{Work Function Calculation}

The work function is obtained from the intercept:
\begin{equation}
    \phi = -c \cdot e = -(-2.143) \times 1.602 = \SI{3.43e-19}{\joule}.
\end{equation}

Converting to electron volts ($\SI{1}{\electronvolt} = \SI{1.602e-19}{\joule}$):

\boxedeq{\phi = \SI{2.14 \pm 0.08}{\electronvolt}}

This agrees with the literature value for potassium: $\phi_{\text{K}} = \SI{2.30}{\electronvolt}$~\cite{michaelson1977} within the combined uncertainty.

% =============================================================================
% 4. ERROR ANALYSIS
% =============================================================================
\section{Error Analysis}

\subsection{Random Uncertainties}

The primary sources of random uncertainty were:
\begin{itemize}
    \item \textbf{Voltage measurement:} The digital voltmeter resolution of \SI{0.1}{\milli\volt} contributed negligible uncertainty compared to measurement variations.
    
    \item \textbf{Repeatability:} The standard deviation of repeated measurements (\SIrange{3}{4}{\milli\volt}) reflects fluctuations in photocurrent and electronic noise.
    
    \item \textbf{Wavelength uncertainty:} Filter bandwidth of $\pm$\SI{5}{\nano\meter} was neglected as mercury lines are narrow ($<$\SI{0.1}{\nano\meter}).
\end{itemize}

\subsection{Systematic Uncertainties}

\begin{enumerate}
    \item \textbf{Contact potential difference:} The work function difference between cathode and anode creates an offset voltage. This affects the intercept (and hence $\phi$) but not the slope (and hence $h$).
    
    \item \textbf{Stray light:} Ambient light entering the phototube would cause spurious photoemission. This was minimized by using a light-tight enclosure.
    
    \item \textbf{Temperature effects:} The work function varies with temperature as $\phi(T) = \phi_0 - \alpha T$. At room temperature, this effect is $<$\SI{0.01}{\electronvolt}.
\end{enumerate}

\subsection{Uncertainty Budget}

Table~\ref{tab:uncertainty} summarizes the uncertainty contributions to the final value of $h$.

\begin{table}[htbp]
\centering
\caption{Uncertainty budget for Planck's constant determination.}
\label{tab:uncertainty}
\begin{tabular}{@{}lccc@{}}
\toprule
Source & Type & $\sigma_h$ (\SI{e-34}{\joule\second}) & \% Contribution \\
\midrule
Voltage repeatability & A & 0.10 & 69\% \\
Frequency calibration & B & 0.05 & 17\% \\
Regression model & A & 0.04 & 11\% \\
Other (stray light, T) & B & 0.02 & 3\% \\
\midrule
\textbf{Combined} & & \textbf{0.12} & \textbf{100\%} \\
\bottomrule
\end{tabular}
\end{table}

Type A uncertainties are evaluated statistically; Type B are estimated from other information.

% =============================================================================
% 5. DISCUSSION
% =============================================================================
\section{Discussion}

\subsection{Comparison with Accepted Value}

The experimental value $h = \SI{6.58 \pm 0.12 e-34}{\joule\second}$ agrees with the CODATA recommended value $h = \SI{6.62607015 e-34}{\joule\second}$~\cite{codata2018} within one standard deviation:
\begin{equation}
    E_n = \frac{|h_{\text{exp}} - h_{\text{acc}}|}{\sigma_h} = \frac{|6.58 - 6.626|}{0.12} = 0.38.
\end{equation}

The normalized error $E_n < 1$ indicates consistency with no evidence of unaccounted systematic errors.

\subsection{Discrepancy in Work Function}

The measured work function $\phi = \SI{2.14}{\electronvolt}$ is approximately 7\% lower than the literature value for clean potassium (\SI{2.30}{\electronvolt}). This discrepancy likely arises from:
\begin{enumerate}
    \item Surface contamination (oxide layer, adsorbed gases)
    \item Contact potential difference not fully corrected
    \item Possible cesium activation of the photocathode
\end{enumerate}

\subsection{Quality of Linear Fit}

The high coefficient of determination ($R^2 = 0.9987$) confirms the linear relationship predicted by Einstein's photoelectric equation. The residuals showed no systematic trend, indicating that higher-order terms are unnecessary.

Figure~\ref{fig:residuals} (not shown) would display residuals versus frequency, confirming homoscedasticity.

% =============================================================================
% 6. CONCLUSIONS
% =============================================================================
\section{Conclusions}

\begin{enumerate}
    \item The photoelectric effect was successfully used to determine Planck's constant with 1.8\% relative uncertainty.
    
    \item The measured value $h = \SI{6.58 \pm 0.12 e-34}{\joule\second}$ is consistent with the accepted value.
    
    \item The work function of the potassium cathode was determined to be $\phi = \SI{2.14 \pm 0.08}{\electronvolt}$.
    
    \item The linear relationship between stopping potential and frequency was verified with $R^2 = 0.9987$.
    
    \item The dominant source of uncertainty was voltage measurement repeatability (69\% of total variance).
\end{enumerate}

\subsection{Recommendations}

Future improvements could include:
\begin{itemize}
    \item Use of a lock-in amplifier to reduce photocurrent noise
    \item Addition of UV wavelengths ($\lambda < \SI{365}{\nano\meter}$) to extend the frequency range
    \item Measurement of the $I$--$V$ characteristic curve for more accurate $V_s$ determination
\end{itemize}

% =============================================================================
% REFERENCES
% =============================================================================
\begin{thebibliography}{9}

\bibitem{einstein1905}
A. Einstein, ``Über einen die Erzeugung und Verwandlung des Lichtes betreffenden heuristischen Gesichtspunkt,'' \textit{Ann. Phys.} \textbf{322}, 132--148 (1905).

\bibitem{michaelson1977}
H. B. Michaelson, ``The work function of the elements and its periodicity,'' \textit{J. Appl. Phys.} \textbf{48}, 4729--4733 (1977).

\bibitem{codata2018}
CODATA Recommended Values of the Fundamental Physical Constants: 2018, NIST. \url{https://physics.nist.gov/cuu/Constants/}

\bibitem{taylor1997}
J. R. Taylor, \textit{An Introduction to Error Analysis}, 2nd ed. (University Science Books, 1997).

\bibitem{bevington2003}
P. R. Bevington and D. K. Robinson, \textit{Data Reduction and Error Analysis for the Physical Sciences}, 3rd ed. (McGraw-Hill, 2003).

\end{thebibliography}

% =============================================================================
% APPENDIX
% =============================================================================
\appendix
\section{Sample Calculations}

\subsection{Frequency from Wavelength}

For $\lambda = \SI{365.0}{\nano\meter}$:
\begin{equation}
    \nu = \frac{c}{\lambda} = \frac{\SI{2.998e8}{\meter\per\second}}{\SI{365.0e-9}{\meter}} = \SI{8.214e14}{\hertz}.
\end{equation}

\subsection{Propagation of Uncertainty}

For a product $h = m \cdot e$, the relative uncertainty is:
\begin{equation}
    \frac{\sigma_h}{h} = \sqrt{\left(\frac{\sigma_m}{m}\right)^2 + \left(\frac{\sigma_e}{e}\right)^2} \approx \frac{\sigma_m}{m},
\end{equation}
since $\sigma_e/e \ll \sigma_m/m$ (the elementary charge is known exactly by definition since 2019).

\vfill
\begin{flushright}
\footnotesize\textit{made with BetterNotes-AI}
\end{flushright}

\end{document}

\documentclass[10pt]{article}

% =========================
% Preamble (matching existing templates)
% =========================
\usepackage[english]{babel}
\usepackage[utf8]{inputenc}
\usepackage[T1]{fontenc}
\usepackage{enumitem}
\usepackage{booktabs}
\usepackage{amsmath}
\usepackage{amsthm}
\usepackage{empheq}
\usepackage{amssymb}
\usepackage{amsfonts}
\usepackage{lmodern}
\usepackage{mathrsfs}
\usepackage{xcolor}
\usepackage{geometry}
\geometry{a4paper, portrait, left = 8mm, right = 8mm, top = 8mm, bottom = 8mm}
\usepackage{hyperref}
\hypersetup{colorlinks=true, linkcolor=blue, urlcolor=cyan}
\usepackage{multicol}
\setlength{\columnseprule}{0.5pt}
\setlength{\columnsep}{0.5cm}

\usepackage{titlesec}
\titlespacing{\section}{0pt}{2pt}{0pt}
\titlespacing{\subsection}{0pt}{2pt}{0pt}
\setlength{\parindent}{0pt}
\pagenumbering{gobble}
\titleformat*{\section}{\large\bfseries}
\titleformat*{\subsection}{\bfseries}

% Colors (Engineering Theme)
\definecolor{probblue}{RGB}{0, 80, 158}
\definecolor{datagreen}{RGB}{0, 128, 64}
\definecolor{resultred}{RGB}{180, 0, 0}
\definecolor{checkpurple}{RGB}{100, 50, 150}

% Custom theorem-like environments
\newtheoremstyle{problemstyle}
	{0pt}{0pt}{\upshape}{}{\bfseries}{}{0.5em}{}
\theoremstyle{problemstyle}
\newtheorem*{problem}{\textcolor{probblue}{Problem}}
\newtheorem*{givendata}{\textcolor{datagreen}{Given}}
\newtheorem*{approach}{\textcolor{probblue}{Approach}}
\newtheorem*{solution}{\textcolor{probblue}{Solution}}

% Result box
\newcommand{\result}[1]{%
\begin{center}
\fcolorbox{resultred}{red!5}{\parbox{0.9\linewidth}{\centering\textbf{\textcolor{resultred}{Result:}} #1}}
\end{center}
}

% Check/verification
\newcommand{\check}[1]{\textcolor{checkpurple}{\textbf{Check:} \textit{#1}}}

% Horizontal rule
\newcommand{\HR}{\vspace{0.15em}\hrule\vspace{0.25em}}

\setlist[itemize]{leftmargin=1.1em,itemsep=0.1em,topsep=0.1em}
\setlist[enumerate]{leftmargin=1.1em,itemsep=0.1em,topsep=0.1em}

\sloppy
\emergencystretch=2em

\begin{document}
\pagestyle{empty}
{\Large\bfseries Fluid Mechanics \& Thermodynamics --- Problem Set \hfill \normalsize Practice Sheet}\HR

\begin{multicols}{2}
\footnotesize

%==========================================================
\section*{1. Bernoulli's Equation in Pipe Flow}
%==========================================================

\begin{problem}
Water flows through a horizontal pipe that narrows from diameter $D_1 = 10$ cm to $D_2 = 5$ cm. The pressure at section 1 is $P_1 = 200$ kPa and velocity is $v_1 = 2$ m/s. Find the pressure $P_2$ at the narrow section.
\end{problem}

\begin{givendata}
\begin{itemize}
\item $D_1 = 0.10$ m, $D_2 = 0.05$ m
\item $P_1 = 200 \times 10^3$ Pa, $v_1 = 2$ m/s
\item $\rho_{water} = 1000$ kg/m$^3$
\item Horizontal pipe $\Rightarrow z_1 = z_2$
\end{itemize}
\end{givendata}

\begin{approach}
Apply \textbf{Continuity} ($A_1 v_1 = A_2 v_2$) and \textbf{Bernoulli} (incompressible, steady, inviscid):
\[
P_1 + \frac{1}{2}\rho v_1^2 + \rho g z_1 = P_2 + \frac{1}{2}\rho v_2^2 + \rho g z_2
\]
\end{approach}

\begin{solution}
\textbf{Step 1:} Cross-sectional areas
\[
A_1 = \frac{\pi D_1^2}{4} = \frac{\pi (0.10)^2}{4} = 7.854 \times 10^{-3} \text{ m}^2
\]
\[
A_2 = \frac{\pi D_2^2}{4} = \frac{\pi (0.05)^2}{4} = 1.963 \times 10^{-3} \text{ m}^2
\]

\textbf{Step 2:} Velocity at section 2 (Continuity)
\[
v_2 = v_1 \frac{A_1}{A_2} = 2 \cdot \frac{7.854}{1.963} = 8 \text{ m/s}
\]

\textbf{Step 3:} Pressure at section 2 (Bernoulli, $z_1=z_2$)
\[
P_2 = P_1 + \frac{1}{2}\rho(v_1^2 - v_2^2) = 200000 + 500(4 - 64)
\]
\[
P_2 = 200000 - 30000 = 170000 \text{ Pa}
\]
\end{solution}

\result{$P_2 = 170$ kPa}

\check{Velocity increased (narrower pipe) $\Rightarrow$ Pressure decreased. Physical sense confirmed.}

\HR
%==========================================================
\section*{2. First Law of Thermodynamics}
%==========================================================

\begin{problem}
A piston-cylinder device contains 0.5 kg of air initially at 300 K and 100 kPa. Heat is added until the temperature reaches 600 K. The process is isobaric. Calculate: (a) Work done by the gas, (b) Heat transfer, (c) Change in internal energy.
\end{problem}

\begin{givendata}
\begin{itemize}
\item $m = 0.5$ kg, $T_1 = 300$ K, $T_2 = 600$ K
\item $P = 100$ kPa (constant, isobaric)
\item Air: $c_p = 1.005$ kJ/(kg$\cdot$K), $c_v = 0.718$ kJ/(kg$\cdot$K)
\item $R_{air} = c_p - c_v = 0.287$ kJ/(kg$\cdot$K)
\end{itemize}
\end{givendata}

\begin{approach}
For isobaric process: $W = P \Delta V = mR\Delta T$. First Law: $Q = \Delta U + W$.
\end{approach}

\begin{solution}
\textbf{(a) Work done:}
\[
W = mR(T_2 - T_1) = 0.5 \times 0.287 \times (600 - 300)
\]
\[
W = 0.5 \times 0.287 \times 300 = 43.05 \text{ kJ}
\]

\textbf{(b) Heat transfer (isobaric $\Rightarrow Q = mc_p\Delta T$):}
\[
Q = mc_p(T_2 - T_1) = 0.5 \times 1.005 \times 300 = 150.75 \text{ kJ}
\]

\textbf{(c) Change in internal energy:}
\[
\Delta U = mc_v(T_2 - T_1) = 0.5 \times 0.718 \times 300 = 107.70 \text{ kJ}
\]
\end{solution}

\result{$W = 43.05$ kJ, $Q = 150.75$ kJ, $\Delta U = 107.70$ kJ}

\check{$Q = \Delta U + W$: $150.75 = 107.70 + 43.05 = 150.75$ ✓}

\HR
%==========================================================
\section*{3. Heat Transfer: Conduction}
%==========================================================

\begin{problem}
A wall of thickness $L = 0.2$ m and thermal conductivity $k = 1.5$ W/(m$\cdot$K) separates two rooms at temperatures $T_1 = 25°$C and $T_2 = -5°$C. The wall area is $A = 10$ m$^2$. Find the heat transfer rate through the wall.
\end{problem}

\begin{givendata}
\begin{itemize}
\item $L = 0.2$ m, $k = 1.5$ W/(m$\cdot$K), $A = 10$ m$^2$
\item $T_1 = 25°$C, $T_2 = -5°$C $\Rightarrow \Delta T = 30$ K
\end{itemize}
\end{givendata}

\begin{approach}
Fourier's Law of heat conduction (1D, steady state):
\[
\dot{Q} = -kA\frac{dT}{dx} = kA\frac{T_1 - T_2}{L}
\]
\end{approach}

\begin{solution}
\[
\dot{Q} = kA\frac{\Delta T}{L} = 1.5 \times 10 \times \frac{30}{0.2} = 2250 \text{ W}
\]
\end{solution}

\result{$\dot{Q} = 2250$ W $= 2.25$ kW}

\check{Heat flows from hot (25°C) to cold ($-5$°C). Direction correct.}

\HR
%==========================================================
\section*{4. Fluid Statics: Hydrostatic Pressure}
%==========================================================

\begin{problem}
A tank contains oil ($\rho_{oil} = 850$ kg/m$^3$) floating on water. The oil layer is 2 m thick, and there is 3 m of water below. Find the gauge pressure at the bottom of the tank.
\end{problem}

\begin{givendata}
\begin{itemize}
\item $\rho_{oil} = 850$ kg/m$^3$, $h_{oil} = 2$ m
\item $\rho_{water} = 1000$ kg/m$^3$, $h_{water} = 3$ m
\item $g = 9.81$ m/s$^2$
\end{itemize}
\end{givendata}

\begin{approach}
Hydrostatic pressure: $P = P_0 + \rho g h$. For layered fluids, sum contributions:
\[
P_{gauge} = \rho_{oil} g h_{oil} + \rho_{water} g h_{water}
\]
\end{approach}

\begin{solution}
\[
P_{gauge} = (850 \times 9.81 \times 2) + (1000 \times 9.81 \times 3)
\]
\[
P_{gauge} = 16677 + 29430 = 46107 \text{ Pa} \approx 46.1 \text{ kPa}
\]
\end{solution}

\result{$P_{gauge} = 46.1$ kPa}

\HR
%==========================================================
\section*{5. Carnot Cycle Efficiency}
%==========================================================

\begin{problem}
A heat engine operates between reservoirs at $T_H = 800$ K and $T_C = 300$ K. The engine receives 1000 kJ of heat per cycle. Calculate: (a) Carnot efficiency, (b) Maximum work output, (c) Heat rejected.
\end{problem}

\begin{givendata}
\begin{itemize}
\item $T_H = 800$ K, $T_C = 300$ K
\item $Q_H = 1000$ kJ
\end{itemize}
\end{givendata}

\begin{approach}
Carnot efficiency: $\eta_C = 1 - \frac{T_C}{T_H}$. Work: $W = \eta_C Q_H$. Rejected heat: $Q_C = Q_H - W$.
\end{approach}

\begin{solution}
\textbf{(a) Carnot efficiency:}
\[
\eta_C = 1 - \frac{T_C}{T_H} = 1 - \frac{300}{800} = 1 - 0.375 = 0.625 = 62.5\%
\]

\textbf{(b) Maximum work output:}
\[
W_{max} = \eta_C \cdot Q_H = 0.625 \times 1000 = 625 \text{ kJ}
\]

\textbf{(c) Heat rejected:}
\[
Q_C = Q_H - W = 1000 - 625 = 375 \text{ kJ}
\]
\end{solution}

\result{$\eta_C = 62.5\%$, $W_{max} = 625$ kJ, $Q_C = 375$ kJ}

\check{$Q_H = W + Q_C$: $1000 = 625 + 375$ ✓}

\HR
%==========================================================
\section*{6. Reynolds Number \& Flow Regime}
%==========================================================

\begin{problem}
Water at 20°C ($\nu = 1.0 \times 10^{-6}$ m$^2$/s) flows through a pipe of diameter 5 cm at velocity 0.5 m/s. Determine the Reynolds number and flow regime.
\end{problem}

\begin{givendata}
\begin{itemize}
\item $D = 0.05$ m, $v = 0.5$ m/s
\item $\nu = 1.0 \times 10^{-6}$ m$^2$/s
\item Transition criteria: Re $< 2300$ laminar, Re $> 4000$ turbulent
\end{itemize}
\end{givendata}

\begin{solution}
\[
\text{Re} = \frac{vD}{\nu} = \frac{0.5 \times 0.05}{1.0 \times 10^{-6}} = \frac{0.025}{10^{-6}} = 25000
\]
\end{solution}

\result{Re $= 25000$ $\Rightarrow$ \textbf{Turbulent flow}}

\HR
%==========================================================
\section*{7. Projectile Motion (Classical Mechanics)}
%==========================================================

\begin{problem}
A ball is thrown with initial velocity $v_0 = 20$ m/s at angle $\theta = 45°$ above horizontal. Find: (a) Maximum height, (b) Range, (c) Time of flight. Neglect air resistance.
\end{problem}

\begin{givendata}
\begin{itemize}
\item $v_0 = 20$ m/s, $\theta = 45°$, $g = 9.81$ m/s$^2$
\item $v_{0x} = v_0 \cos\theta = 14.14$ m/s
\item $v_{0y} = v_0 \sin\theta = 14.14$ m/s
\end{itemize}
\end{givendata}

\begin{solution}
\textbf{(a) Maximum height:}
\[
H = \frac{v_{0y}^2}{2g} = \frac{(14.14)^2}{2 \times 9.81} = \frac{200}{19.62} = 10.2 \text{ m}
\]

\textbf{(b) Range:}
\[
R = \frac{v_0^2 \sin(2\theta)}{g} = \frac{400 \times 1}{9.81} = 40.8 \text{ m}
\]

\textbf{(c) Time of flight:}
\[
T = \frac{2v_{0y}}{g} = \frac{2 \times 14.14}{9.81} = 2.88 \text{ s}
\]
\end{solution}

\result{$H = 10.2$ m, $R = 40.8$ m, $T = 2.88$ s}

\HR
%==========================================================
\section*{8. Work-Energy Theorem}
%==========================================================

\begin{problem}
A 2 kg block slides down a frictionless incline of height 5 m. Find its speed at the bottom using conservation of energy.
\end{problem}

\begin{solution}
Conservation of mechanical energy: $E_i = E_f$
\[
mgh + 0 = 0 + \frac{1}{2}mv^2 \quad \Rightarrow \quad v = \sqrt{2gh}
\]
\[
v = \sqrt{2 \times 9.81 \times 5} = \sqrt{98.1} = 9.9 \text{ m/s}
\]
\end{solution}

\result{$v = 9.9$ m/s (or $\sqrt{2gh}$ in general)}

% made with BetterNotes-AI
\begin{flushright}\footnotesize \textit{made with BetterNotes-AI}\end{flushright}

\end{multicols}

\end{document}
